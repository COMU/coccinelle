\documentclass[22pt]{article}
\usepackage[turkish]{babel}
\usepackage[utf8]{inputenc}
\usepackage{multicol}
\usepackage{listings}
\setlength{\columnsep}{3cm}

\begin{document}
\textbf{14 Alıştırmalar}\\
\\
Bu bölümde bir dizi örnek sunuyor. Her bir örnek, uygulandığı bir C kodu ile birlikte sunulmaktadır. Kuralların tanımı ve eşleştirme işlemi açıklanır.\\
\\
\textbf{14.1 Fonksiyonu Yeniden Adlandırma}\\
\\
Coccinelle temel amaçlarından biri de yazılım evrimini yerine getirmektir. Örneğin, fonksiyonu yeniden adlandırmayı gerçekleştirmek için Coccinelle kullanılabilir. Aşağıdaki örnekte, her olayda çağırılan foo fonksiyonu, bar fonksiyonu olarak yeniden değiştirilir.\\

\begin{multicols*}{3}
				
\begin{lstlisting}
once
#DEFINE TEST "foo"		

print("foo");			

int main(int i) {		
//Test				
  int k = foo();		

  if(1) {			
    foo();  			
  } else {			
    foo();			
  }				

  foo();			
}
\end{lstlisting}

\columnbreak

\begin{lstlisting}
anlamsal yama		

@@			

@@			

			
-foo()			
+bar()
\end{lstlisting}
\columnbreak
\begin{lstlisting}
	sonra

#DEFINE TEST "foo"

print("foo");

int main(int i) {
//Test
  int k = bar();

  if(1) {
    bar();
  } else {
    bar();	
  }

  bar();
}
\end{lstlisting}
\end{multicols*}
\end{document}
