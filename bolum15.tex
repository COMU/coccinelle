\documentclass[22pt]{article}
\usepackage[turkish]{babel}
\usepackage[utf8]{inputenc}
\usepackage[a4paper,left=2cm,right=2cm,top=2cm,bottom=2cm]{geometry}
\usepackage{listings}
\begin{document}
\textbf{15 İpuçları ve hileler}\\
\\
Eğer bir fonksiyon çağrısı ile bir işaretçi değerinin herhangi bir erişimini yeniden yazmak istiyorsanız, aşagıdaki anlamsal yamayı kullanabilirsiniz.
\begin{lstlisting}
1 - a = *b
2 + a = readb(b)
\end{lstlisting}
Ancak, bazı nedenlerden dolayı kodunuz "bar=*(foo)" benziyorsa,  "bar=readb((foo))" sona erdirecekseniz, "foo" etrafında ekstra parantezi meta değer "b" ile yakalayabilirsiniz. İyi çıkış kodu üretmek için aşağıdaki anlamsal yamayı kullanabilirsiniz.
\begin{lstlisting}
1 - a = *(b)
2 + a = readb(b)
\end{lstlisting}
Ve güvenilir standart .iso izomorfizm dosyaları içermelidir:
\begin{lstlisting}
1 Expression
2 @ paren @
3 expression E;
4 @@
5
6 (E) => E
\end{lstlisting}
"bar = *(foo)" ve "bar = *foo "  concinelle eşdeğer kabul eder ve her ikisi de yakalayacaktır. Son olarak, "bar = ready(for)" beklendiği gibi oluşturur.
\end{document}