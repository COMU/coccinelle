\documentclass[a4paper,20pt, right=2cm]{article}
\usepackage[turkish]{babel}
\usepackage[utf8]{inputenc}
\usepackage[T1]{fontenc}
\usepackage[cm]{fullpage}
\usepackage{listings}
\begin{document}
\textbf{13 Yineleme}\\
\\
Bir sonraki yineleme sanal kuralları veya sanal tanımlayıcı bağlamalarının farklı bir kümesi verilerek, Coccinelle de yineleme yapmak mümkündür. demos/iteration.cocci 'de bir örneği bulunmaktadır. Örnek aşağıdaki gibidir:
\begin{lstlisting}
	virtual after_start

	@initialize:ocaml@

	let tbl = Hashtbl.create(100)

	let add_if_not_present from f file =
	try let _ = Hashtbl.find tbl (f,file) in ()
	with Not_found ->
	    Hashtbl.add tbl (f,file) file;
	    let it = new iteration() in
	    (match file with
	        Some fl -> it#set_files [fl]
	    | None -> ());
	    it#add_virtual_rule After_start;
	    it#add_virtual_identifier Err_ptr_function f;
	    it#register()
\end{lstlisting}
after\_start sanal kuralı ilk yinelemeyi (burada eşleşme kabul edilmez) diğerlerinden ayırt etmek için kullanılır. This is done by not mentioning after\_start in the command line, ama her yinelemede eklenir.\\
\\
"new iteration" ve "register" satırları arasında bulunan "add\_if\_not\_present" fonksiyonu ana kod için çalışan yinelemedir. "new iteration" yeni bir yinelemeyi temsil eden bir yapı oluşturur. "set\_files" yeni yineleme üzerinde dikkate alınması gereken dosyaların listesini belirler. Eğer fonsiyon çağırılmazsa yeni yineleme o dosyalara o an ki güncel yinelemymiş gibi davranır. "add\_virtual\_rule A" yanına -D parametresi koyulduğunda da aynı etkiyi yaratacaktır. Bu başka bir yerde yapılmasa da kural adının büyük harfle başladığını unutmayın. "add\_virtual\_identifier X v" yanına -D x=v parametresi koyulduğunda da aynı etkiyi yaratacaktır. Durum değişikliği hakkında not, "extend\_virtual\_identifiers()" güncel yinelemenin "add\_virtual\_identifier" tarafından üzerien yazılmamış tüm sanal tanımlarını korur. Son olarak, "register" gelecekte bir zamanda yeni bir yinelemeyi tetiklemek için toplanan bilgileri kuyruğa sokar.\\
\\
Yineleme kullanılırken değişikliğe izin verilmez. Böylelikle anlamsal yama *s (anlamsal eşleme yerine bir anlamsal yama.) içermedikçe "-no-show-diff" kullanmayı gerektirir. Kodun geri kalanı üzerinde aynı bilginin birden fazla kez kuyruğa alınmadığından emin olmak için bir anahtarlı tablo(hash table) kullanır. Coccinelle'nin kendisi bunu desteklemez.\\
\end{document}