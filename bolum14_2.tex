\documentclass[22pt]{article}
\usepackage[turkish]{babel}
\usepackage[utf8]{inputenc}
\usepackage{listings}
\usepackage{url}

\begin{document}
\textbf{14.2 Fonksiyon Argümanı Kaldırmak}\\
\\
Bir başka önemli değişim çeşidi fonksiyon argümanı giriş ve silme işlemleridir. Aşağıdaki örnekte, rule1 kuralında dönüş tipi ve 2 parametresi olan bir fonksiyon tanımlanmış. İkinci isimsiz kural üç argümana sahip önceden eşleşmiş olan fonksiyonu çağırır. Sonrasında üçüncü argüman yeni fonksiyon prototipine uyması için kaldırılır.\\
\\

\begin{lstlisting}
@ rule1 @
identifier fn;
identifier irq, dev_id;
typedef irqreturn_t;
@@
static irqreturn_t fn (int irq, void *dev_id)
{
...
}
@@
identifier rule1.fn;
expression E1, E2, E3;
@@
fn(E1, E2
- ,E3
  )
\end{lstlisting}
dönüşümden önce \path{drivers/atm/firestream.c} dosyasındaki 1653. satır\\
\begin{lstlisting}
static void fs_poll (unsigned long data)
{
struct fs_dev *dev = (struct fs_dev *) data;
fs_irq (0, dev, NULL);
dev->timer.expires = jiffies + FS_POLL_FREQ;
add_timer (&dev->timer);
}
\end{lstlisting}
dönüşümden sonra \path{drivers/atm/firestream.c} dosyasındaki 1653.satır\\
\begin{lstlisting}
static void fs_poll (unsigned long data)
{
struct fs_dev *dev = (struct fs_dev *) data;
fs_irq (0, dev);
dev->timer.expires = jiffies + FS_POLL_FREQ;
add_timer (&dev->timer);
}
\end{lstlisting}
\end{document}
